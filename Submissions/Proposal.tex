% Options for packages loaded elsewhere
\PassOptionsToPackage{unicode}{hyperref}
\PassOptionsToPackage{hyphens}{url}
\documentclass[
  11pt,
]{article}
\usepackage{xcolor}
\usepackage[margin=1in]{geometry}
\usepackage{amsmath,amssymb}
\setcounter{secnumdepth}{-\maxdimen} % remove section numbering
\usepackage{iftex}
\ifPDFTeX
  \usepackage[T1]{fontenc}
  \usepackage[utf8]{inputenc}
  \usepackage{textcomp} % provide euro and other symbols
\else % if luatex or xetex
  \usepackage{unicode-math} % this also loads fontspec
  \defaultfontfeatures{Scale=MatchLowercase}
  \defaultfontfeatures[\rmfamily]{Ligatures=TeX,Scale=1}
\fi
\usepackage{lmodern}
\ifPDFTeX\else
  % xetex/luatex font selection
\fi
% Use upquote if available, for straight quotes in verbatim environments
\IfFileExists{upquote.sty}{\usepackage{upquote}}{}
\IfFileExists{microtype.sty}{% use microtype if available
  \usepackage[]{microtype}
  \UseMicrotypeSet[protrusion]{basicmath} % disable protrusion for tt fonts
}{}
\makeatletter
\@ifundefined{KOMAClassName}{% if non-KOMA class
  \IfFileExists{parskip.sty}{%
    \usepackage{parskip}
  }{% else
    \setlength{\parindent}{0pt}
    \setlength{\parskip}{6pt plus 2pt minus 1pt}}
}{% if KOMA class
  \KOMAoptions{parskip=half}}
\makeatother
\usepackage{graphicx}
\makeatletter
\newsavebox\pandoc@box
\newcommand*\pandocbounded[1]{% scales image to fit in text height/width
  \sbox\pandoc@box{#1}%
  \Gscale@div\@tempa{\textheight}{\dimexpr\ht\pandoc@box+\dp\pandoc@box\relax}%
  \Gscale@div\@tempb{\linewidth}{\wd\pandoc@box}%
  \ifdim\@tempb\p@<\@tempa\p@\let\@tempa\@tempb\fi% select the smaller of both
  \ifdim\@tempa\p@<\p@\scalebox{\@tempa}{\usebox\pandoc@box}%
  \else\usebox{\pandoc@box}%
  \fi%
}
% Set default figure placement to htbp
\def\fps@figure{htbp}
\makeatother
\setlength{\emergencystretch}{3em} % prevent overfull lines
\providecommand{\tightlist}{%
  \setlength{\itemsep}{0pt}\setlength{\parskip}{0pt}}
\usepackage[]{natbib}
\bibliographystyle{unsrtnat}
\usepackage{bookmark}
\IfFileExists{xurl.sty}{\usepackage{xurl}}{} % add URL line breaks if available
\urlstyle{same}
\hypersetup{
  pdftitle={STA380 Project Proposal: Monte Carlo Simulation of a Hypothesis Test of Binomially Distributed Samples},
  pdfauthor={Siling Cheng, Junyi Hou, Valerie Pan, Tianhong Shen, Feiyang Xue},
  hidelinks,
  pdfcreator={LaTeX via pandoc}}

\title{STA380 Project Proposal: Monte Carlo Simulation of a Hypothesis
Test of Binomially Distributed Samples}
\author{Siling Cheng, Junyi Hou, Valerie Pan, Tianhong Shen, Feiyang
Xue}
\date{2026-02-11}

\begin{document}
\maketitle

\subsection{Project Topic}\label{project-topic}

Monte Carlo Simulation of a Hypothesis Test of Binomially Distributed
Samples {[}One-sample z-test for a binomial proportion{]}

\subsection{Simulation vs.~Dataset}\label{simulation-vs.-dataset}

We will using pure simulation

\subsection{Project Detail}\label{project-detail}

Hypothesis tests for binomial data are widely used in applications
involving binary outcomes, such as quality control and clinical trials.
Monte Carlo simulation allows practitioners to assess whether the normal
approximation performs adequately for their specific sample size and
parameter settings.

\[
X \sim \text{Binomial}(n, p), \qquad
\] We'll consider the following Hypothesis: \[
H_0: p = p_0 \quad \text{vs.} \quad H_a: p \neq p_0
\] To estimate Type I error, Type II error, and power, we will be
following Monte Carlo procedure\citep{robertCasella2004}:

\begin{itemize}
  \item Repeatedly generate binomial samples under a specified parameter value $p$.
  \item For each simulated sample, perform a normal approximation test and record the resulting p-value.
  \item Use the empirical distribution of p-values to approximate probabilities related to the test's performance.
\end{itemize}

The following outputs will be provided:

\begin{itemize}
  \item A histogram of simulated p-values, with a vertical line indicating the chosen significance level $\alpha$.
  \item A ggplot of the power curve
  \item A table showing Type I error rate, Type II error rate and power of the test
  \item For more visualization, we would provide a  ggplot of the power curve, a plot of type I and II error trade-off.

\end{itemize}

\subsection{User Input(Shiny
Components)}\label{user-inputshiny-components}

\textless\textless\textless\textless\textless\textless\textless{} HEAD n
\textless- 15 p \textless- 0.3 n\_sim \textless-10000 set.seed(234)
sim\_results \textless- rbinom(n\_sim, size = n, prob = p) sim\_data
\textless- as.data.frame(table(sim\_results) / n\_sim)
colnames(sim\_data) \textless- c(``Successes'', ``Proportion'')
sim\_data\(Successes <- as.numeric(as.character(sim_data\)Successes)
theoretical\_x \textless- 0:n theoretical\_y \textless-
dbinom(theoretical\_x, size = n, prob = p) theoretical\_data \textless-
data.frame(x = theoretical\_x, y = theoretical\_y) ggplot() +
geom\_col(data = sim\_data, aes(x = Successes, y = Proportion), fill =
``lightblue'', color = ``white'', alpha = 0.7) + geom\_line(data =
theoretical\_data, aes(x = x, y = y), color = ``red'', size = 1) +
geom\_point(data = theoretical\_data, aes(x = x, y = y), color =
``red'', size = 2) + scale\_x\_continuous(breaks = 0:n) + labs(title =
paste(``Monte Carlo Simulation (n ='', n, ``, p ='', p, ``, Samples ='',
n\_sim, ``)''), x = ``Number of Successes'', y = ``Probability'') +
theme\_minimal()

======= Users will be able to modify the following:

\begin{quote}
\begin{quote}
\begin{quote}
\begin{quote}
\begin{quote}
\begin{quote}
\begin{quote}
fe0f75f5d10b8605f7c6384bd799a7fd687c172e \#\#\# Core Simulation Controls
1. \textbf{Simulation Sample Size}: - Users can set the number of Monte
Carlo simulations (range: 1,000 to 10,000) via a dual slider or numeric
input. - Demonstrates how increasing the sample size stabilizes Type I
error and power estimates, reducing simulation-based variance. 2.
\textbf{Random Seed}: - An integer input field to initialize the
pseudo-random number generator. - Ensures reproducibility in scientific
research, allowing users to generate identical results for verification
and ecological modeling. 3. \textbf{Number of Trials (\(n\))}: - A
slider (range: 10--500) representing the sample size of individual
units. - Affects the denominator of the z-score; illustrates how larger
\(n\) reduces standard error and improves test sensitivity. 4.
\textbf{Success Probability (\(p\))}: - A slider (range: 0 to 1) for the
true population proportion. - Allows users to explore how different
effect sizes impact the statistical power of the test.
\end{quote}
\end{quote}
\end{quote}
\end{quote}
\end{quote}
\end{quote}
\end{quote}

\subsubsection{Advanced Features}\label{advanced-features}

\begin{enumerate}
\def\labelenumi{\arabic{enumi}.}
\setcounter{enumi}{4}
\tightlist
\item
  \textbf{Significance level (\(\alpha\)):}

  \begin{itemize}
  \tightlist
  \item
    Users can select from a few significance levels.
  \item
    Dynamically updates the rejection region on the plot and
    recalculates the estimated Type I error rate.
  \end{itemize}
\item
  \textbf{Colour Themes:}

  \begin{itemize}
  \tightlist
  \item
    Users can select from a few colour scheme options (including
    palettes from \citep{nichols2026}) and change the colour of the plot
    features (e.g.~bars or lines).
  \item
    Enhances accessibility for color-blind users and allows for clear
    visual distinction between plot elements.
  \end{itemize}
\item
  \textbf{Hypothesis Test Directionality:}

  \begin{itemize}
  \tightlist
  \item
    A toggle switch or dropdown box between \textbf{one-sided} and
    \textbf{two-sided} tests.
  \item
    Users can select a view among a one-tailed test and two-tailed test.
    This shows how directional hypotheses change results and p-values
    without altering the sample size or effect size.
  \end{itemize}
\end{enumerate}

\subsection{Reference}\label{reference}

\citep{shiny2025} \citep{R-base} \citep{ggplot22026} \citep{nichols2026}

\subsection{Generative AI Usage
Statement}\label{generative-ai-usage-statement}

Generative AI was used to help format BibTeX references, as well as
providing tips on using R Markdown and GitHub (e.g.~solving merge
conflicts).

\bibliography{references.bib}

\end{document}
